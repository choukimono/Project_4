\documentclass[a4paper, twoside, 11pt]{report}

\usepackage[utf8]{inputenc}
\usepackage[T1]{fontenc}
\usepackage[english]{babel}
\usepackage[top= 120pt, left=80pt, right=80pt]{geometry} %marges
\usepackage{setspace} %interlignage
\usepackage{url}
\usepackage{graphicx}
\usepackage{lmodern} 
\usepackage{array}
\usepackage{csquotes}
\usepackage[numbers,square]{natbib}
\usepackage{soul}
\usepackage{hyperref}
\usepackage{amsthm}
\usepackage{color}
\usepackage[usenames,dvipsnames,svgnames,table]{xcolor}
\usepackage{adjustbox}
\usepackage{amssymb}
\usepackage{amsmath}
\usepackage{dsfont}
%%\usepackage{braket}
\usepackage{physics}
\usepackage{amsfonts}
\usepackage[numbers,square]{natbib}
\usepackage{multirow}
\usepackage{listings}
\usepackage{wasysym}

%%%%%% COULEURS CODE C++
\lstdefinestyle{customc}{
  belowcaptionskip=1\baselineskip,
  breaklines=true,
  frame=L,
  xleftmargin=\parindent,
  language=C,
  showstringspaces=false,
  basicstyle=\footnotesize\ttfamily,
  keywordstyle=\bfseries\color{red},
  commentstyle=\itshape\color{gray},
  identifierstyle=\color{NavyBlue},
  stringstyle=\color{black},
}

\lstdefinestyle{customasm}{
  belowcaptionskip=1\baselineskip,
  frame=L,
  xleftmargin=\parindent,
  language=[x86masm]Assembler,
  basicstyle=\footnotesize\ttfamily,
  commentstyle=\itshape\color{purple!40!black},
}

\lstset{escapechar=@,style=customc}

%%%%%% STYLES DE THÉORÈMES

\newtheoremstyle{theorem}%	Name
  {}%	Space above
  {}%	Space below
  {}%	Body font
  {}%	Indent amount
  {\bfseries}%	Theorem head font
  {.}%	Punctuation after theorem head
  { }%	Space after theorem head, ' ', or \newline
  {}%	Theorem head spec (can be left empty, meaning `normal')

\newtheoremstyle{exemple}%	Name
  {}%	Space above
  {}%	Space below
  {\color{Gray}\itshape}%	Body font
  {}%	Indent amount
  {\color{Gray}\itshape}%	Theorem head font
  {.}%	Punctuation after theorem head
  { }%	Space after theorem head, ' ', or \newline
  {}%	Theorem head spec (can be left empty, meaning `normal')

\newtheoremstyle{remark}%	Name
  {}%	Space above
  {}%	Space below
  {\itshape}%	Body font
  {}%	Indent amount
  {\bfseries}%	Theorem head font
  {.}%	Punctuation after theorem head
  { }%	Space after theorem head, ' ', or \newline
  {}%	Theorem head spec (can be left empty, meaning `normal')
  
%%%%%% DÉCLARATION DES THÉORÈMES

\theoremstyle{theorem}
\newtheorem{theorem}{Theorem}[section]
\newtheorem{lemme}{Lemma}[section]
\newtheorem{proposition}{Proposition}[section]
\newtheorem{definition}{Definition}[section]

\theoremstyle{remark}
\newtheorem{remark}{Remark}[chapter]

\theoremstyle{exemple}
\newtheorem*{exemple}{Example}


%%%%%% COMMANDES QUI SIMPLIFIENT LA VIE

\newcommand{\legende}[1]{
\begin{center}
	\begin{minipage}{12cm}
		\begin{center}
			\textit{\textcolor{WildStrawberry!30}{#1}}
		\end{center}
	\end{minipage}
\end{center}}

\newcommand{\sherlock}[2]{
	\begin{equation}
		\textcolor{WildStrawberry}{#1}
	\end{equation}
	\legende{#2}
}
	
\newcommand{\sherlocked}[1]{
	\begin{equation}
		\textcolor{WildStrawberry}{#1}
	\end{equation}
}

	
\newcommand{\defSherlock}[3]{
	\begin{definition}[\textbf{#1}]
		\sherlock{#2}{#3}
	\end{definition}
}

\newcommand{\defSherlocked}[2]{
	\begin{definition}[\textbf{#1}]
		\sherlocked{#2}
	\end{definition}
}	

\newcommand{\propSherlock}[3]{
	\begin{proposition}[\textbf{#1}]
		\sherlock{#2}{#3}
	\end{proposition}
}

\newcommand{\propSherlocked}[2]{
	\begin{proposition}[\textbf{#1}]
		\sherlocked{#2}
	\end{proposition}
}

\newcommand{\textSherlocked}[1]{
	\begin{center}
		\textcolor{WildStrawberry}{#1}
	\end{center}
}

\newcommand{\mimi}{\mathrm{Je\ t'aime}}

\newcommand{\N}{\mathbb{N}}
\newcommand{\Z}{\mathbb{Z}}
\newcommand{\R}{\mathbb{R}}
\newcommand{\C}{\mathbb{C}}



%%%%%%%%%%%%%%%%%%%%%%%%%%%%%%%%%%%%%%%%%%%%%%%%%%%%%%%
%%%%%%%%%%%%%%%%%%%%%%%%%%%%%%%%%%%%%%%%%%%%%%%%%%%%%%%
%%%%%%%%%%%%%%%%%%%%%%%%%%%%%%%%%%%%%%%%%%%%%%%%%%%%%%%

\title{FYS3150\\Project 4 - }
\author{Ethel Villeneuve}
\date{October 2017 \\University of Oslo \\ \url{https://github.com/choukimono/Project_4.git}}



\begin{document}
\selectlanguage{english}
\maketitle
	
	
\begin{abstract}
    
    \paragraph{}
    
\end{abstract}


\tableofcontents


\chapter*{Introduction}
\addcontentsline{toc}{chapter}{Introduction}

    \paragraph{}
    

\chapter{Theory}
    
    \section{The Ising model}
    
        \subsection{The general model}
        
            \paragraph{}The Ising model is a mathematical model used in statistical mechanics. It consists of discrete variables, which represent the magnetic moment of the spin, which can take the value $+1$ or $-1$. The spins will only interact with their direct neighbors. With this model, we can study the phase transitions at finite temperature for magnetic systems. We can expressed the energy as 
                
                \begin{equation*}
                    E = -J \sum\limits_{<kl>}^{N}s_ks_l - B \sum\limits_{k}^{N}s_k
                    \tag{1}
                \end{equation*}
            %Mets pas de saut de ligne ici patate, ça évitera l'alinéa moche.
            with $J$ a constant expressing the strength of the interaction between the neighboring spins, $<kl>$ indicating the fact that we only sum the nearest neighbors, $N$ the number of spins, $s_{k,l} = \pm 1$ and $B$ an external magnetic field interacting with the magnetic moment set up by the spins. \\
            
            This is the general expression of the Ising model. For our use, we will only focus on the case where $B=0$. \\
            Then, we will be able to calculate expectation values of the mean energy $<E>$ and magnetization $<M>$ at a given temperature. To do this, we will use a Boltzmann distribution
            
                \begin{equation*}
                    P_i(\beta) = \frac{e^{-\beta E_i}}{Z}
                \end{equation*}
             %blblbl
            where $P_i$ is the probability of finding the system in the state $i$, $\beta = \frac{1}{kT}$, $T$ being the temperature and $k$ the Boltzmann constant, $E_i$ is the energy of a state $i$ and $Z$ is the partition function defined by $\displaystyle Z = \sum\limits_{i=1}^{M}e^{-\beta E_i}$ with $M$ the number of states. $E_i$, which is the energy in the state $i$, is given by 
            
                \begin{equation*}
                    E_i = -J \sum\limits_{<kl>}^{N}s_ks_l 
                    \tag{2}
                \end{equation*}
            with $k,l$ the different spins of the state $i$.
                
            A simple particular case of the Ising model, the two-dimensional square lattice model, allows us to have an analytical solution. 
        
        \subsection{Two-dimensional square lattice model}
        
            \paragraph{}This particular case is one of the simplest models to show a phase transition. It is defined by conditions : the external magnetic field $B=0$, this is a two-dimensional lattice with N sites, with periodic boundary conditions. Let's take the case with $N=2\times2$ spins. We have $2^4 = 16$ different states with those conditions. We reuse the equation (2) to find the energy of each configuration with the spins-up taking the value $+1$ and the spins-down taking the value $-1$. Let's take for example the case where three spins are pointing up (and so one is pointing down) numbered from 1 to 4 :
                $\begin{matrix}
                    \downarrow ^{(1)} & \uparrow ^{(2)}\\
                    \uparrow ^{(3)} & \uparrow ^{(4)}
                \end{matrix}$.
                
                \begin{align*}
                    E &= -J \sum\limits_{<kl>}^{4}s_ks_l \\
                      &= -J (s_1s_2 + s_2s_1 + s_1s_3 + s_3s_1 + s_2s_4 + s_4s_2 + s_3s_4 + s_4s_3) \\
                      &= -J \left[(-1) + (-1) + (-1) + (-1) + 1 + 1 + 1 + 1\right] = -J \times 0 \\
                    E &= 0 
                \end{align*}
                
            The magnetization formula is $\displaystyle M = \sum\limits_{k}^{N} s_k$. So in our example, we have 
                
                \begin{align*}
                    M &= s_1 + s_2 + s_3 + s_4 \\
                      &= (-1) + 1 + 1 + 1 \\
                    M &= 2
                \end{align*}
            
            The following table sums up all the possible states of a two-dimensional square lattice model.
            
                \begin{center}
                    \begin{tabular}{|*{8}{c|}}
                            \hline
                        Number of spins-up & \multicolumn{4}{c|}{Possible configurations} & Degeneracy & Energy & Magnetization \\
                            \hline
                            \hline
                        $4$ & \multicolumn{4}{c|}{$\begin{matrix}
                                   \uparrow & \uparrow \\
                                   \uparrow & \uparrow
                                \end{matrix}$} & $1$ & $-8J$ & $4$ \\
                            \hline
                        $3$ & $\begin{matrix}
                            \downarrow & \uparrow \\
                            \uparrow & \uparrow
                        \end{matrix}$ & $\begin{matrix}
                                            \uparrow & \downarrow \\
                                            \uparrow & \uparrow
                                         \end{matrix}$ & $\begin{matrix}
                                                               \uparrow & \uparrow \\
                                                               \downarrow & \uparrow
                                                           \end{matrix}$ & $\begin{matrix}
                                                                                \uparrow & \uparrow \\
                                                                                \uparrow & \downarrow
                                                                            \end{matrix}$ & $4$ & $0$ & $2$ \\
                            \hline
                        $2$ & $\begin{matrix}
                                   \uparrow & \uparrow \\
                                   \downarrow & \downarrow
                               \end{matrix}$ & $\begin{matrix}
                                                     \downarrow & \downarrow \\
                                                     \uparrow & \uparrow
                                                \end{matrix}$ & $\begin{matrix}
                                                                    \uparrow & \downarrow \\
                                                                    \uparrow & \downarrow
                                                                 \end{matrix}$ & $\begin{matrix}
                                                                     \downarrow & \uparrow \\
                                                                     \downarrow & \uparrow
                                                                \end{matrix}$ & $4$ & $0$ & $0$ \\
                            \hline
                        $2$ & \multicolumn{2}{c|}{$\begin{matrix}
                                    \uparrow & \downarrow \\
                                    \downarrow & \uparrow
                                \end{matrix}$} & \multicolumn{2}{c|}{$\begin{matrix}
                                                     \downarrow & \uparrow \\
                                                     \uparrow & \downarrow
                                                  \end{matrix}$} & $2$ & $8J$ & $0$ \\
                            \hline
                        $1$ & $\begin{matrix}
                            \downarrow & \downarrow \\
                            \downarrow & \uparrow
                        \end{matrix}$ & $\begin{matrix}
                                            \downarrow & \downarrow \\
                                            \uparrow & \downarrow
                                         \end{matrix}$ & $\begin{matrix}
                                                               \downarrow & \uparrow \\
                                                               \downarrow & \downarrow
                                                           \end{matrix}$ & $\begin{matrix}
                                                                                \uparrow & \downarrow \\
                                                                                \downarrow & \downarrow
                                                                            \end{matrix}$ & $4$ & $0$ & $-2$ \\
                            \hline
                        $0$ & \multicolumn{4}{c|}{$\begin{matrix}
                                                       \downarrow & \downarrow \\
                                                       \downarrow & \downarrow
                                                     \end{matrix}$} & $1$ & $-8J$ & $-4$ \\
                            \hline            
                    \end{tabular}
                    Table : Energy and magnetization for a $N = 2\times 2$-spin Ising model with periodic boundary conditions.
                \end{center}
                
    \section{}
    
    
    \section{}



\chapter{Implementation}


\chapter{Results}


\chapter*{Conclusion}
\addcontentsline{toc}{chapter}{Conclusion}

    \paragraph{}
    

\chapter*{Bibliography}
    
    \begin{itemize}
        \item
    \end{itemize}
    

\end{document}